%-----------------------------------------------------------------
%	ABSTRACT
%	!TEX root = ./../main.tex
%-----------------------------------------------------------------
\cleardoubleevenemptypage
\thispagestyle{empty}
\phantomsection
\addcontentsline{toc}{section}{Abstract}
\begin{abstract}
	% \begin{enumerate}[(a)]
	% 	\item Introduction. In one sentence, what’s the topic?
	% 	\item State the problem you tackle.
	% 	\item Summarize (in one sentence) why nobody else has adequately answered the research question yet.
	% 	\item Explain, in one sentence, how you tackled the research question.
	% 	\item In one sentence, how did you go about doing the research that follows from your big idea.
	% 	\item As a single sentence, what’s the key impact of your research?
	% \end{enumerate}
	The influence of global warming on the intensity of tropical-cyclones (TC, also called hurricanes or typhoons in an abuse of language) is a rather controversial topic that has already been addressed in many statistical studies, such as~\textcite{Webster2005}.
	For the North-Atlantic basin, it has been shown~\cite{Corral2010} that the probability distribution of the so-called power-dissipation index ($PDI$, a rough estimation of released energy) is indeed affected by the anual and basin-wide averaged sea surface temperature (SST), displacing towards more extreme values on warm years (high SST).
	As the $PDI$ integrates (cubic) wind speed over tropical-cyclone lifetime, it is an open question where the $PDI$ increase comes from (higher speed, longer lifetime, or both).

	Our empirical results show a remarkable correlation in the joint distribution of lifetime and $PDI$, and linear regression of the logarithmic variables yields a power-law relation between both. Statistical testing, by means of a permutation test, shows that this relation does not significantly depend on the SST. In other words, the wind speed of a tropical cyclone of a given lifetime will be the same (within statistical fluctuations) in cold and warm years.
	Nevertheless, the increase of TC lifetime with SST triggers an increase in wind speed as a by-product. Further analysis shows that the longer lifetimes are mainly due to a shift to South-East of the TC genesis point.

	Our conclusions are compatible with the view of tropical cyclones as an activation process, in which, once the event has started, its intensity is kept in critical balance between attenuation and intensification (and so, higher SST does not trigger more intensification).
	Summarizing, storms with the same lifetime should have the same speed and $PDI$, no matter the SST, as, once the cyclone is activated, the wind speed at each TC stage should not depend on the SST.
\end{abstract}
