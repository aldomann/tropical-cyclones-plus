%-----------------------------------------------------------------
%	INTRODUCTION: TROPICAL-CYCLONES
%	!TEX root = ./../main.tex
%-----------------------------------------------------------------
\subsection{Tropical-cyclones}\label{sec:intro-hurricanes}
To characterise a tropical-cyclone one needs to define a physically relevant measure of released energy. In \cite{Emanuel1986-bis} \citeauthor{Emanuel1986-bis} showed that in a tropical-cyclone energy dissipation occurs mostly in the atmospheric surface layer, and that the corresponding dissipation rate per unit area, $D$, is
\begin{align}
	D \equiv C_{D} \rho \norm{\va{v}}^{3}
	,
\end{align}
where $C_{D}$ is the surface drag coefficient, $\rho$ is the surface air density, $\norm{\va{v}}$ is the magnitude of the surface wind velocity.

Thus, integrated over the surface area covered by a circularly symmetric tropical-cyclone of radius $r_{0}$ of lifetime $\tau$, the total power dissipated by the storm, $PD$, is given by
\begin{align}
	PD \equiv 2 \pi \int_{0}^{\tau} \int_{0}^{r_{0}} C_{D} \rho \norm{\va{v}}^{3} r \dd{r} \dd{t}
	.
\end{align}

However, as stated in \cite{Emanuel2005}, since the integral in the expression \eqref{eq:pdi} will in practice be dominated by high wind speeds, one can approximate the product $C_{D} \rho$ as a constant and define a simplified power dissipation index ($PDI$) as
\begin{subequations}
\begin{align}\label{eq:pdi}
	PDI \equiv \int_{0}^{\tau} v_{max}^{3} \dd{t}
	.
\end{align}

However, the wind data (as we discuss in~\cref{ssec:hurdat}) is recorded every $\Delta t = \SI{6}{\hour}$. Therefore, we discretise the expression for the $PDI$ using the rectangle method:
\begin{align}\label{eq:pdi-bis}
	PDI = \sum_{t} v_{t}^{3} \Delta t
	,
\end{align}
\end{subequations}
where $v_{t}$ is the maximum sustained surface wind speed at time $t$.

\medskip
Although the $PDI$ is enough to characterise the released energy by a tropical-cyclone, we are interested in the causal relationship between increasing tropical-cyclone $PDI$ and increasing sea surface temperature (SST) proposed by \textcite{Trenberth2005}, as well as the relationship with the lifetime associated to each tropical-cyclone.

\citeauthor{Trenberth2005} states that higher SSTs are associated with increased water vapour in the lower troposphere; both higher SSTs and increased water vapour tend to increase the energy available for atmospheric convection and the energy available to tropical-cyclones as a consequence.

By separating the $PDI$ data by low-SST and high-SST years, we get two similar $PDI$ distributions with one major difference: high-SST years should have a longer right tail on account of having more available energy from the sea (this is explored with more detail in \Cref{ssec:univariate}). For this separation (or classification) process we follow the methodology used by \citeauthor{Corral2010} in~\cite{Corral2010}, which is a variation of the methodology proposed by \citeauthor{Webster2005} in~\cite{Webster2005}.

\sk
To perform this classification we need to calculate first the mean SST of each year $\ev{\text{SST}}$:
\begin{align}
	\ev{\text{SST}} = \sum_{y} \frac{\text{SST}(y)}{Y},
\end{align}
where $\text{SST}(y)$ is the mean SST of the year $y$, and $Y$ is the total number of years studied; as usual, the standard error of this mean, is defined as
\begin{align}\label{eq:sst-sem}
	\se{\text{SST}} = \frac{1}{\sqrt{Y}} \sqrt{ \frac{1}{Y-1} \sum_{y} \qty(\text{SST}(y) - \ev{\text{SST}})^{2} } .
\end{align}

The classification of each year in low-SST and high-SST years is performed depending on whether they are lower or greater than $\ev{\text{SST}}$.

\medskip
If one wishes to consult a thorough technical review article on tropical-cyclone as a thermodynamic system,~\cite{Emanuel2003} by \citeauthor{Emanuel2003} would probably be the best source available.

% To characterise a tropical-cyclone (TC) one needs to define a physically relevant measure of released energy. The released energy of each TC is summarised as
% \begin{equation}\label{eq:pdi}
% 	PDI = \sum_{t} v_{t}^{3} \Delta t .
% \end{equation}
% The raw hurricane best track data (HURDAT2) is provided by the National Hurricane Center. We intentionally limit this study to the satellite era (1966--2016), as it is the most reliable.

% \bigskip
% Then, the hurricane observational data is classified into occurrences in low-SST and high-SST years depending on whether they are lower or greater than
% \begin{equation}
% 	\ev{\text{SST}} = \sum_{y \in Y} \frac{\text{SST}(y)}{Y} ,
% \end{equation}
% where $\text{SST}(y)$ is the mean SST of the year $y$, and $Y$ is the total number of years studied. The SST data (HadISST1) is provided by the Met Office Hadley Centre.

\bigskip
Tropical cyclones are classified into three main groups, based on wind intensity: tropical depressions, tropical storms, and a third group of more intense storms, whose name depends on the region. In particular, in the Northeast Pacific or in the North Atlantic, it is called a hurricane. In \Cref{tab:hurr-class} we can see a detailed classification of the tropical-cyclones studied in this text.
\begin{table}[H]
	\centering
	\begin{tabular}{l c c}
		\toprule
		\toprule
		\multicolumn{1}{c}{Category} & \multicolumn{2}{c}{1--minute sustained winds} \\
		\midrule
		Tropical depression  & $\le \SI{33}{\knot}$      & ($\le \SI{61}{\km/\hour}$) \\
		Tropical storm       & $\SIrange{34}{63}{\knot}$ & ($\SIrange{63}{118}{\km/\hour}$) \\
		Category 1 hurricane & $\SIrange{64}{82}{\knot}$ & ($\SIrange{119}{153}{\km/\hour}$) \\
		Category 2 hurricane & $\SIrange{83}{95}{\knot}$ & ($\SIrange{154}{177}{\km/\hour}$) \\
		Category 3 major hurricane & $\SIrange{96}{112}{\knot}$  & ($\SIrange{178}{208}{\km/\hour}$) \\
		Category 4 major hurricane & $\SIrange{113}{136}{\knot}$ & ($\SIrange{209}{253}{\km/\hour}$) \\
		Category 5 major hurricane & $\ge \SI{137}{\knot}$       & ($\ge \SI{254}{\km/\hour}$) \\
		\bottomrule
	\end{tabular}
	\caption{Tropical-cyclone classification used by the NHC}
	\label{tab:hurr-class}
\end{table}
