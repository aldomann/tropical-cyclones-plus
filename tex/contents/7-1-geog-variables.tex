%-----------------------------------------------------------------
%	GEOGRAPHIC ANALYSIS
%	!TEX root = ./../main.tex
%-----------------------------------------------------------------

% \begin{itemize}
% 	\item Distance
% 	\item Linear regression?
% 	\item Analysis of initial/final position.
% 	\item Marginals of the finitial positions (4 plots per basin + summary).
% 	\item Boxplots and Wilcoxon tests.
% 	\item Clustering.
% \end{itemize}

\subsection{Geographical variables}
During the analysis performed in \Cref{sec:data-analysis} and \Cref{sec:regr-analysis-boot} we only used geographical information about the hurricane occurrences for the calculation of the basin-wide averaged SST.

From the raw HURDAT2 data sets, we can extract some relevant geographical data about each tropical-cyclone that could help obtain some insight or physical reason behind the displacement of the joint distribution of $PDI$ and lifetime of the storms between low-SST and high-SST years occurrences.

\bigskip
We focus on the geographical genesis location of a tropical-cyclone, as well as its death location. This information is easy to get from the individual longitude and latitude tracks for each hurricane.

Apart from this, we calculate the total travelled distance, or path length, of each hurricane by means of the \emph{spherical law of cosines}:
\begin{align}
	d(p_{1}, p_{2}) = \cos^{-1}\qty[\sin(\phi_{1}) \cdot \sin(\phi_{2}) + \cos(\phi_{1}) \cdot \cos(\phi_{2}) \cdot \cos(\Delta \lambda)] \cdot R_{\text{E}},
\end{align}
where $\phi$ and $\lambda$ respectively represent latitude and longitude in radians, and $R_{\text{E}}$ represents the Earth's radius in meters.

\medskip
In \Cref{hd:geog-data-head} one can see the structure of the cleaned data illustrate these newly calculated geographical variables.
\begin{table}[H]
	\centering
	\ttfamily
	\resizebox{\textwidth}{!}{%
	\begin{tabular}{r r r r c r r r r r}
		\toprule
		\toprule
		storm.id & storm.name & n.obs & storm.duration & \multirow{2}{*}{$\cdots$} & first.lat & last.lat & first.long & last.long &  distance \\
		<chr>    & <chr>      & <int> &          <dbl> &          &     <dbl> &    <dbl> &      <dbl> &     <dbl> &     <dbl> \\
		\midrule
		AL011966 & ALMA       &    42 &         907200 & $\cdots$ &      12.7 &     42.0 &      -84.0 &     -70.5 &   4129705 \\
		AL021966 & BECKY      &     9 &         194400 & $\cdots$ &      32.4 &     45.5 &      -57.8 &     -58.5 &   1678458 \\
		AL031966 & CELIA      &    36 &         777600 & $\cdots$ &      19.1 &     52.0 &      -59.5 &     -57.0 &   5635625 \\
		AL041966 & DOROTHY    &    37 &         799200 & $\cdots$ &      31.0 &     53.5 &      -41.0 &     -38.5 &   2864172 \\
		AL051966 & ELLA       &    26 &         561600 & $\cdots$ &      10.0 &     24.3 &      -35.0 &     -68.4 &   3917529 \\
		AL071966 & GRETA      &    26 &         561600 & $\cdots$ &      13.7 &     28.0 &      -48.4 &     -71.7 &   2977350 \\
		\bottomrule
	\end{tabular}}
	\caption{Excerpt of the North Atlantic data set, focusing on the geographical variables}
	\label{hd:geog-data-head}
\end{table}

\bigskip
Similarly to the unified HURDAT2 \& HadISST1 data sets (see \Cref{ssec:unified-data-set}), these geographically enhanced data sets are available in the \texttt{HurdatHadISSTData} package:
\begin{itemize}
	\item \texttt{tc.pdi.geog.natl} -- Data set for the North Atlantic basin.
	\item \texttt{tc.pdi.geog.epac} -- Data set for the Northeast Pacific basin.
	\item \texttt{tc.pdi.geog.all} -- Data set for both basins.
\end{itemize}
