%-----------------------------------------------------------------
%	BOOTSTRAP: CONCLUSIONS
%	!TEX root = ./../main.tex
%-----------------------------------------------------------------
\section{Conclusions}
In this work we have thoroughly explored and analysed the theoretical foundation of the linear regression under the ordinary least squares method. In particular, we saw both graphically and numerically how some of the assumptions on the random error, such as homoscedasticity and normality, do not hold for the joint distribution of $PDI$ and storm lifetime for the North Atlantic and Northeast Pacific basins.

To solve these problems, we used bootstrap to resample the observations in order to provide a more accurate and robust regression analysis than the one provided by the OLS method.

Last but not least, we proposed a statistical test to compare low-SST and high-SST years by performing a permutation test. This allowed us to quantify the statistical significance of evidence against the hypothesis that storms of equal lifetime have the same $PDI$ and same joint distribution, regardless of the SST. The results provide strong evidence that this hypothesis is indeed true, as none of the performed tests rejected the null hypothesis.

Our conclusions are compatible with the view of tropical cyclones as an activation process, in which, once the event has started, its intensity is kept in critical balance between attenuation and intensification (and so, higher SST does not trigger more intensification).

\bigskip
An open question, nonetheless, is why the increase of tropical-cyclone lifetime with SST triggers an increase in wind speed as a by-product.

The results of a simple exploratory analysis of the geographical properties of the tropical-cyclones show that the longer lifetimes for high-SST are mainly due to a shift to South-East of the tropical-cyclones genesis point, although further analysis is needed.

The steps to follow would be to perform a hierarchical clustering of the location of genesis and death of storms using the aggregate information provided by the $PDI$, lifetime, location, and path length of each storm to have a deeper understanding of the difference between hurricane occurrences in low-SST and high-SST years.
